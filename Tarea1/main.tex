%%%%%%%%%%%%%%%%%%%%%%%%%%%%% Define Article %%%%%%%%%%%%%%%%%%%%%%%%%%%%%%%%%%
\documentclass{article}


\usepackage{tcolorbox}

\begin{document}
\begin{center}
    {\Large\textsc{Tarea 1}}\\
    {\small\textsc{Pregunta 1: Lógica proposicional}}\\
    \small{\textsc{Matemáticas Discretas}}
\end{center}

Integrantes:
\begin{itemize}
    \item Franco Cattani
    \item Nicolás del Valle
    \item Jorge Espinosa
\end{itemize}

\begin{enumerate}
    \item Suponga que le entregan un algoritmo de \textit{caja negra}
    de resolución SAT, es decir, un dispositivo que toma una fórmula de 
    lógica proposicional $\phi$ y devuelve si $\phi$ es o no satisfacible.
    Usted no sabe nada sobre el funcionamiento de este algoritmo. Vamos a 
    denotar este algoritmo $A$, por lo que $A(\phi)$ es verdadero si $\phi$
    es satisfacible.

    Esta pregunta plantea qué más se puede hacer con un algoritmo de 
    resolución SAT.
    \begin{enumerate}
        \item Cree un algoritmo que utilice $A$ como subrutina para determinar 
         si $\phi$ es una tautología. Demuestre que su algoritmo es correcto. 
         No se limite a enumerar todas las posibles asignaciones y comprobar cada una individualmente.

        Respuesta: \\ 
            Para este algoritmo utilizaremos el siguiente \textit{lemma}

        \begin{tcolorbox}[title=\textit{Lemma}]
            La proposición $\phi$ es una tautología si y solo sí $\neg\phi$ es no 
            satisfacible.
        \end{tcolorbox}
         
        En base a esto el algoritmo se le pasa como parametro $\phi$ para luego utilizar como subrutina 
        $A$ con el parámetro $\neg\phi$, es decir, $A(\neg\phi)$, si $A(\neg\phi)=F$ entonces $\phi$
         es tautología, de lo contrario no es tautología.

         \item Suponga que tiene dos fórmulas proposicionales $\phi$ y $\psi$. Te interesa determinar 
         si $\phi \equiv \psi$, es decir, si $\phi$ y $\psi$ tienen siempre los mismos valores de verdad.
        Crea un algoritmo que utilice $A$ como subrutina para responder esta pregunta, y demuestra que tu 
        algoritmo es correcto.
        
        Respuesta:

        
    \end{enumerate}
\end{enumerate}

\end{document}