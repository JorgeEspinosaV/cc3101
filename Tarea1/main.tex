%%%%%%%%%%%%%%%%%%%%%%%%%%%%% Define Article %%%%%%%%%%%%%%%%%%%%%%%%%%%%%%%%%%
\documentclass[10pt]{article}
\usepackage[utf8]{inputenc}
\usepackage[spanish]{babel}
\usepackage[dvipsnames,table]{xcolor}
\usepackage{graphicx}
\usepackage{float}
\usepackage[a4paper, total={6in, 8in}]{geometry}
\usepackage{tcolorbox}

\begin{document}

\begin{center}
    {\Large\textsc{Tarea 1}}\\
    {\small\textsc{Pregunta 1: Lógica proposicional}}\\
    \small{\textsc{Matemáticas Discretas}}
\end{center}

Integrantes:
\begin{itemize}
    \item Franco Cattani
    \item Nicolás del Valle
    \item Jorge Espinosa
\end{itemize}

\begin{enumerate}
    \item Suponga que le entregan un algoritmo de \textit{caja negra}
    de resolución SAT, es decir, un dispositivo que toma una fórmula de 
    lógica proposicional $\phi$ y devuelve si $\phi$ es o no satisfacible.
    Usted no sabe nada sobre el funcionamiento de este algoritmo. Vamos a 
    denotar este algoritmo $A$, por lo que $A(\phi)$ es verdadero si $\phi$
    es satisfacible.

    Esta pregunta plantea qué más se puede hacer con un algoritmo de 
    resolución SAT.
    \begin{enumerate}
        \item Cree un algoritmo que utilice $A$ como subrutina para determinar 
         si $\phi$ es una tautología. Demuestre que su algoritmo es correcto. 
         No se limite a enumerar todas las posibles asignaciones y comprobar cada una individualmente.

        \textbf{\underline{Respuesta:}} Para este algoritmo utilizaremos el siguiente \textit{lemma}

        \begin{tcolorbox}[title=\textit{Lemma}]
            La proposición $\phi$ es una tautología si y solo sí $\neg\phi$ es no 
            satisfacible.
        \end{tcolorbox}
         
        En base a esto el algoritmo se le pasa como parametro $\phi$ para luego utilizar como subrutina 
        $A$ con el parámetro $\neg\phi$, es decir, $A(\neg\phi)$, si $A(\neg\phi)=F$ entonces $\phi$
         es tautología, de lo contrario no es tautología.

         \item Suponga que tiene dos fórmulas proposicionales $\phi$ y $\psi$. Te interesa determinar 
         si $\phi \equiv \psi$, es decir, si $\phi$ y $\psi$ tienen siempre los mismos valores de verdad.
        Crea un algoritmo que utilice $A$ como subrutina para responder esta pregunta, y demuestra que tu 
        algoritmo es correcto.
        
        \textbf{\underline{Respuesta:}}

        Al algoritmo se le pasarán dos parámetros, $\phi$ y $\psi$, luego se guardará el resultado de 
        $A(\neg(\phi\equiv\psi))$, este valor de verdad lo llamaremos $\zeta $, si $\zeta=Falso$  entonces 
        $\phi$ y $\psi$ siempre tienen el mismo valor de verdad, si por el contrario $\zeta=Verdadero$ 
        entonces $\phi$ y $\psi$ no siempre tienen el mismo valor de verdad. Con esto el algoritmo cumple su
        función el cual es comprobar la equivalencia entre $\phi$ y $\psi$.
        
        Para demostrar esto nos basaremos en el \textit{lemma} mencionado anteriormente, sabemos que 
        $\phi\equiv\psi$ es una fórmula proposicional, entonces tiene un valor de verdad. 
        \clearpage 
        Para poder determinar si se cumple esta equivalencia $\phi\equiv\psi$ debe ser una tautología, 
        recordemos que la definición de tautología es;

        \begin{tcolorbox}[title=\textit{Tautología}]
            Una fórmula $\alpha\in\xi(P)$ es una \textbf{tautología} si y solo sí es \textbf{verdadera} bajo
            cualquier valuación, es decir, si y solo sí $\widehat{\sigma}(\alpha)=1$ para toda valuación
            $\sigma:P\rightarrow \{0,1\}$  
        \end{tcolorbox}

        La definición de tautología para este caso nos dice que para toda valuación $P$, $\phi\equiv\psi$ 
        siempre se cumplirá, por ende, $\zeta=Falso$ es lo mismo que decir que $\neg(\phi\equiv\psi)$ es 
        insatisfacible si y solo si $\phi\equiv\psi$ es una tautología, con lo que el algoritmo cumple con
        lo pedido.
        \item Suponga que tiene una fórmula proposicional $\phi$ con $n$ variables que sabe que es satisfacible. 
        Cree un algoritmo que utilice $A$ como subrutina para obtener una asignación satisfactoria para $\phi$ 
        utilizando como máximo $n$  llamadas a $A$. Demuestre que su respuesta es correcta.

        El algoritmo tomará como variables $\phi , n , A$, supongamos que las $n$ variables son \newline 
        $P:\{P_1,P_2,...,P_n\}$, se guardarán todas las $n$ variables de $\phi$, para luego recorrerlas. No se 
        les asignarán una valuación inicialmente.
        
        Se le asignará una valuacion ${0,1}$ a $P_i$ (primer caso $P_0$), con esto quedarán $n-i$ variables sin 
        valuación, ahora se le pasará $A(\phi ^i)$ el cuál corresponde a $\phi$ con las $i$ asignaciones 
        ya dadas, si $A(\phi^i)=\texttt{Falso}$ entonces se cambiará el valor dado a $P_i$ y se pasará a $P_{i+1}$.
        Así hasta llegar a $i=n$ donde se entregarán las asignaciones contenidas en $\phi^n$. 

        Notemos que $\phi$ es satisfacible por ende existe una asignación satisfactoria, si la asignación $P_i$ no 
        satisface $\phi$ entonces $\neg P_i$ si lo hará, ya que, sabemos que $\phi$ es satisfacible. Así sucesivamente
        con las $n$ variables. Con esto demostramos que el algoritmo entregará una asignación satisfactoria y además 
        es fácil notar  tiene $n$ llamadas a $A$ con lo que cumple con lo pedido.
        

        
    \end{enumerate}
    \item En esta pregunta usted construirá fórmulas de la lógica proposicional que definen como sumar números
    binarios.

    Considere el conjunto de variables proposicionales \newline 
    $P=\{a_0, a_1, ... , a_{n-1},b_0, b_1, ..., b_{n-1}\}$.
    Podemos suponer que cada valuación $\sigma : P\rightarrow\{0,1\}$ define un par de números binarios $X^{\sigma}_1$ 
    y $X^{\sigma}_2$ dados por la evaluación de $\sigma$ sobre las secuencias de variables $a_{n-1}...a_1a_0 $ y 
    $b_{n-1}...b_1b_0$.
    
    Por ejemplo, si $n=3$ y $\sigma$ es tal que, $\sigma(a_2)=\sigma(a_1)=\sigma(b_1)=0$ y 
    $\sigma(a_0)=\sigma(b_2)=\sigma(b_0)=1$ entonces $X^{\sigma}_1=001$ y $X^{\sigma}_2=101$.

    Construya fórmulas (solamente con los conectivos $\{\vee ,\neg\}$) $\phi_0,\phi_1,..,\phi_{n-1},\phi_n$ en $\mathcal{L}(P)$
    tales que para toda valuación $\sigma$ se cumpla que $\sigma(\phi_i)=1$, si y solo si, el bit en la posición $i$
    de $X^{\sigma}_1+X^{\sigma}_2$ es 1.

    \textit{Sus fórmulas las deben dejar en función de $\{ \neg, \vee\}$ , en caso de que usen algún otro conectivo
    lógico $(\{ \wedge ,\Rightarrow,\oplus, etc\})$, \textbf{DEBEN} definirlo en base a $\{ \neg, \vee \}$, por más trivial que sea 
    (esto es enunciado solamente)}.

    Para poder definir la suma de números binarios debemos cubrir 2 situaciones las cuales son:
    \begin{enumerate}
        \item Suma de bits para cada $i-esima$ posición, es decir, $a_i$ + $b_i$.
        \item Cuando $a_i+b_i+c_{i-1}>1$ con $c_{i-1}$ que representa al "bit sobrante" de la suma de $a_{i-1}+b_{i-1}+c_{i-2}$.
    \end{enumerate}

    Notemos que en (b) tenemos un problema para $i=0$, por ende, para $i=0$ tomamos $c_{i_{0}}=0$. Comenzamos definiendo 
    $c_{i-1}$: Claramente $c_{i-1}:\{0,1\}$ por lo tanto para que este sea 1, suponiendo que $c_{i-2}=1$,
    $a_{i}=1$ ó $b_{i}=1$, por otro lado si $c_{i-2}=0$ obliga a $a_i=1$ y $b_i=1$, por último para que $c_{i-1}=0$ obliga tanto a 
    $a_i=0$ y $b_i=0$, por tanto la fórmula proposicional que define lo enunciado es:


    \begin{center}
        $\mathcal{C}(a_i, b_i, c_{i-1})=(a_i\wedge b_i)\vee(a_i\wedge c_{i-1})\vee(b_i\wedge c_{i-1})$ 
    \end{center}
    Para dejar esta fórmula en los conectivos $\{\neg, \vee\}$ basta con aplicar ley de morgan.
    \begin{center}
        $\mathcal{C}(a_i, b_i, c_{i-1})=\neg(\neg a_i\vee \neg b_i)\vee\neg(\neg a_i\vee \neg c_{i-1})\vee\neg(\neg b_i\vee \neg c_{i-1})$ \\
    \end{center}

    Para la suma de bits basta con el operador $\oplus$ $(XOR)$ el cual debemos expresar en los conectivos $\{\neg,\vee\}$.
    \begin{center}
        $a_i\oplus b_i\oplus c_{i-1}\equiv (a_i\oplus b_i)\oplus c_{i-1}\equiv a_i\oplus(b_i\oplus c_{i-1})$\\
        $\equiv (\{(a_i\wedge\neg b_i)\vee(\neg a_i\wedge b_i)\}\wedge \neg c_{i-1})\vee(\{\neg(a_i\vee\neg b_i)\vee\neg(\neg a_i\wedge b_i)\}\wedge c_{i-1})$
    \end{center}
    Nuevamente para dejar esta fórmula solo con los conectivos indicado más arriba, basta con aplicar ley de morgan.
    \begin{center}
        $a_i\oplus b_i\oplus c_{i-1}\equiv\neg(\neg\{\neg(\neg a_i\vee b_i)\vee\neg(a_i\vee\neg b_i)\}\vee c_{i-1})\vee\neg(\{\neg(\neg a_i\vee b_i)\vee\neg(a_i\vee\neg b_i)\}\vee\neg c_{i-1})$
    \end{center}
    Con estas dos podemos definir las fórmulas:
    \begin{itemize}
        \item $\phi_0=a_0\oplus b_0 \oplus c_{i_0}$
        
        \hspace{0.55cm}$\vdots$

        \item $\phi_i=a_i\oplus b_i \oplus \mathcal{C}(a_{i-1},b_{i-1},c_{i-2})$
        
        \hspace{0.55cm}$\vdots$

        \item $\phi_{n-1}=a_{n-1}\oplus b_{n-1}\oplus\mathcal{C}(a_{n-2},b_{n-2},c_{n-3})$
        \item $\phi_n=\mathcal{C}(a_{n-1},b_{n-1},c_{n-2})$
    \end{itemize}

    $\mathcal{C}(a_i,b_i,c_{i-1})=c_i$ es el acarreamiento (bit sobrante de la suma de bits de la $i-esima$ posición).
        
    Con esto podemos sumar números binarios donde tendremos dos fórmulas lógicas las que nos definen la operación suma.
    
    \end{enumerate}



\end{document}